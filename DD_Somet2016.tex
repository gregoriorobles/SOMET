%%%%%%%%%%%%%%%%%%%%%%%%%% author.tex %%%%%%%%%%%%%%%%%%%%%%%%%
%
% sample root file for your contribution to an IFIP volume
% published at Springer
%
% Use this file as a template for your own input.
%
%%%%%%%%%%%%%%%%%%%%%%%% Springer-Verlag %%%%%%%%%%%%%%%%%%%%%%%%%%


% RECOMMENDED %%%%%%%%%%%%%%%%%%%%%%%%%%%%%%%%%%%%%%%%%%%%%%%%%%%
\documentclass[ifip]{svmult}

% choose options for [] as required from the list
% in the Reference Guide, Sect. 2.2
\usepackage{epsfig} 
\usepackage{makeidx}         % allows index generation
\usepackage{graphicx}        % standard LaTeX graphics tool
                             % when including figure files
\usepackage{multicol}        % used for the two-column index
\usepackage[bottom]{footmisc}% places footnotes at page bottom
% etc.
% see the list of further useful packages
% in the Reference Guide, Sects. 2.3, 3.1-3.3

\makeindex             % used for the subject index
                       % please use the style sprmidx.sty with
                       % your makeindex program


%%%%%%%%%%%%%%%%%%%%%%%%%%%%%%%%%%%%%%%%%%%%%%%%%%%%%%%%%%%%%%%%%%%%%

\begin{document}

\title*{Towards traceability between issues and reports for code review time analysis.}
% Use \titlerunning{Short Title} for an abbreviated version of
% your contribution title if the original one is too long
\author{Dorealda Dalipaj\inst{1}}
% Use \authorrunning{Short Title} for an abbreviated version of
% your contribution title if the original one is too long
\institute{Universidad Rey Juan Carlos, Madrid.
\texttt{dorealda.dalipaj@urjc.es}
}
%
% Use the package "url.sty" to avoid
% problems with special characters
% used in your e-mail or web address
%
\maketitle

\textbf{\textit{Abstract --}} Researchers working with software repositories, 
often when building performance or quality models, need to recover traceability links between bug reports 
in issue tracking repositories and reviews in code review systems. 
\\
However, too often the information stored in bug tracking repositories is not
explicitly tagged or linked to the issues reviewing them. Researchers have to
resort to heuristics to tag the data (e.g., to identify if an issue is a bug report or a work item), 
or to link a piece of code to a particular issue or bug.
\\
Recent studies argue that software models based on imperfect datasets, with missing
links to the code and incorrect tagging of issues, display biases
that compromise the validity and generality of the models built on top of the datasets. 
This risk enhances furthermore the importance of the task of recovering 
traceability links between reports and issues. 
\\
In this study we propose two different automatic approaches on how to link report and issue 
for an open source cloud computing project that enforces 
strict development guidelines and rules on the quality of the 
data in its issue tracking repository. We empirically compare the outcome of the two 
approaches and use the most prominent one to quantitatively analyse two important metrics of performance 
of code review in software maintenance: time to identify bug reports and time to carry out the review process.
\\
\\
\textit{\textbf{Keywords:} traceability, code review, open source software, quantitative analysis.}


\section{Introduction}
\label{sec:1}
% Always give a unique label
% and use \ref{<label>} for cross-references
% and \cite{<label>} for bibliographic references
% use \sectionmark{}
% to alter or adjust the section heading in the running head
Software process data, especially bug reports and code changes, are widely used in software engineering research. The
integration of reports and reviews, on one hand, provides valuable information on the history and evolution of a software project 
\cite{con0}. 
On the other hand, it provide the means for conducting studies that qualitatively and quantitatively analyse fundamental parameters 
for characterization of quality and performance of the code review and the most important metrics having a positive increasing 
relation with the benefits of this process.

It is used, e.g., to build models which study the impact of characteristics of issue and review discussions on the 
defect-proneness of a patch \cite{con1}, to spot defect-introducing code changes during review, to predict the number and locale 
of bugs in future software releases, or to investigate technical and non-technical 
factors influencing modern code review ( \cite{con2, con3, con4, con5, con6}).

However, to achieve such studies, 
much of the information stored in software repositories
must be processed and cleaned for further analysis. Such 
processing and cleaning is often done using assumptions 
and heuristics. A generally accepted assumption is that most issue
tracking systems contain just bug reports. Actually, 
they contain other items such as approved design specifications for 
additions and changes to the project team’s code repositories or 
lightweight feature specifications.

The quality of these assumptions and heuristics represents a threat to 
the validity of results derived from software 
repositories. Recent studies, by Bird et al. \cite{con7} and Antoniol et al. 
\cite{con8}, have casted doubts on the quality of data produced
using such heuristics. 
Bird et al. \cite{con7} note that when considering links between bug reports and actual 
code, there exists bias in the severity level of bugs and the experience of
developers fixing bugs, thus a quality model built using only the linked 
bugs, will be biased towards the behaviour of more experienced developers.
Antoniol et al. \cite{con8} note that many of the issues in an issue tracking system do 
not actually represent bug reports, therefore, using such data might lead to 
incorrect bug counts for the different parts of a software system.

To prevent or minimize such situations, we would require a perfect dataset where such linkages 
and tagging of issues was done by professional developers and with great attention to detail. 
While such a perfect dataset might not exist, we believe that a near-ideal dataset does exist. 
This dataset is derived from the OpenStack project. The OpenStack project follows a disciplined 
software development process with careful attention to continuously maintaining the linkages 
between issues and code changes through automated tool support. The strong discipline and the 
automated tool support gives us strong confidence that this project represents a near-ideal 
case of high quality linkage dataset that is correctly tagged.

Furthermore we used this dataset to quantitatively analyse the responsiveness of code review in 
measuring the amount of time that developers spend to individuate bugs and in how much they spend 
for fixing an issue.
\\
\\
Summarizing, this paper provides the following two contributions:
\begin{itemize}
 \item[$\bullet$] we present two fully automated approaches for linking issues from 
 bugtracking system to reviews from code review system of Open Stack,
 \item[$\bullet$] we then use the dataset obtained to quantitatively analyse two of the 
 most fundamental metrics of the responsiveness for the code review process of Open Stack. 
\end{itemize}

In the remainder of this paper, we first describe the necessary background notions for our work (section 3). Next, we describe the 
case study setup (section 4), then present the results (section 5). After threats to 
validity and future work (section 6 and section 7), we discuss some conclusions (section 8).

\section{Background}
\label{sec:2}

This section provides background information about previous research areas that motivated this study, 
environments that we focus on and tools for obtaining data from their repositories. 
\\
Areas closely related to this research include data extraction and integration, data quality 
in software engineering, 
data verification in software repositories, and data quality effects on empirical software 
engineering.
\\
Software engineering process data such as bug reports and
version control log files are widely used in empirical software
engineering. Therefore, the extraction and integration of this data is critical.
Fischer et al. \cite{con0} presented a Release History Database which contains 
the version control log and the bug report information. To link the change log and 
the bug tracking database, Fischer et al. searched for change log messages which 
match to a given regular expression. Later, they improved the linking algorithm and 
built in a file-module verification \cite{con9}. A similar approach to link the 
change log with the bug tracking database was chosen by other researchers. All of 
them used regular expressions to find bug report link candidates in the change log 
file (e.g., \cite{con10, con6, con11, con12, con13}).
\\
Liebchen and Shepperd \cite{con14} surveyed hundreds of empirical software engineering 
papers to assess how studies manage data quality issues. They found only 23 that 
explicitly referenced data quality. Four of the 23 suggested that data quality 
might impact analysis, but made no suggestion of how to deal with it. They conclude that 
there is very little work to assess the quality of data sets and point to the extreme 
challenge of knowing the \textit{true} values and populations. They suggest that 
simulation-based approaches might help.
\\
Bettenburg et al. \cite{con15, con16, con17} provided first analysis of bug report quality. 
They investigated the attributes of a good bug report surveying developers and used it to 
develop a computational model of a bug report quality. The resulting model allowed to 
display the current quality of a defect report whilst typing.
\\
Hooimeijer et al. \cite{con18} also analyzed the quality of defect reports and tried to 
predict whether the defect report will be closed within a given amount of time.
Chen et al. \cite{con19} studied the change logs of three Open Source projects and analyzed 
the quality of these log files. 
\\
Bachmann et al. \cite{con20} surveyed five Open Source and one Closed Source project 
in order to provide a deeper insight into the quality and characteristics of these 
often-used process data. Specifically, they defined quality and characteristics measures, 
computed them and discussed the issues that arose from these observations. 
They showed that there are vast diferences between the projects, particularly with respect 
to the quality of the link rate between bugs and commits.
\\
Aranda et al. \cite{con21} provided a field study of coordination activities around bug fixing, 
based on a survey of software professionals at Microsoft. Specifically, they studied 10 bugs in 
detail and showed that: (i) electronic repositories often hold incomplete or incorrect data, 
and (ii) the histories of even simple bugs are strongly dependent on social, organizational, and 
technical knowledge that cannot be solely extracted through the automated analysis of software
repositories. They report that software repositories show an incomplete picture of the social 
processes in a project.
\\
Our approach uses a reverse engineered model of the code review process of Open Stack, and extracts 
key information from the issue-tracking and code review systems. 

%\subsection{Subsection Heading}
%\label{sec:2}
%Your text goes here.

%\begin{equation}
%\vec{a}\times\vec{b}=\vec{c}
%\end{equation}

%\subsubsection{Subsubsection Heading}
%Your text goes here. Use the \LaTeX\ automatism for cross-references as
%well as for your citations, see Sect.~\ref{sec:1}.

%\paragraph{Paragraph Heading} 
% Your text goes here.

%\subparagraph{Subparagraph Heading.} Your text goes here.%
%
%\index{paragraph}
% Use the \index{} command to code your index words
%
% For tables use
%
%\begin{table}
%\centering
%\caption{Please write your table caption here}
%\label{tab:1}       % Give a unique label
%
% For LaTeX tables use
%
%\begin{tabular}{lll}
%\hline\noalign{\smallskip}
%first & second & third  \\
%\noalign{\smallskip}\hline\noalign{\smallskip}
%number & number & number \\
%number & number & number \\
%\noalign{\smallskip}\hline
%\end{tabular}
%\end{table}
%
%
% For figures use
%
%\begin{figure}
%\centering
% Use the relevant command for your figure-insertion program
% to insert the figure file.
% For example, with the option graphicx use
%\includegraphics[height=4cm]{figure.eps}
%
%\caption{Please write your figure caption here}
%\label{fig:1}       % Give a unique label
%\end{figure}
%
% For built-in environments use
%
%\begin{theorem}
%Theorem text\footnote{Footnote} goes here.
%\end{theorem}
%
% or
%
%\begin{lemma}
%Lemma text goes here.
%\end{lemma}
%
%
% BibTeX users please use
% \bibliographystyle{}
% \bibliography{}
%
% Non-BibTeX users please follow the syntax
% the syntax of "referenc.tex" for your own citations
%%%%%%%%%%%%%%%%%%%%%%%% referenc.tex %%%%%%%%%%%%%%%%%%%%%%%%%%%%%%
% sample references
% "computer science"
%
% Use this file as a template for your own input.
%
%%%%%%%%%%%%%%%%%%%%%%%% Springer-Verlag %%%%%%%%%%%%%%%%%%%%%%%%%%

%
% BibTeX users please use
% \bibliographystyle{}
% \bibliography{}
%
% Non-BibTeX users please use
\begin{thebibliography}{99.}
%
% and use \bibitem to create references.
%
% Use the following syntax and markup for your references
%
\bibitem{con0} M. Fischer, M. Pinzger, and H. Gall. Populating a release history database from version 
control and bug tracking systems. Proceedings of the International Conference on Software Maintenance 
(ICSM 03). IEEE Computer Society, 2003.

\bibitem{con1} The Impact of Human Discussions on Just
-In-Time Quality Assurance. Parastou Tourani and Bram Adams. SANER 2016 - 
23rd IEEE International Conference on Software Analysis, Evolution, and Reengineering.

\bibitem{con2}  A. Bernstein, J. Ekanayake, and M. Pinzger. Improving defect prediction 
using temporal features and non linear models. In IWPSE'07, pages 11-18, Dubrovnik, Croatia,
September 2007.

\bibitem{con3} Baysal, O., Kononenko, O., Holmes, R., \& Godfrey, M. W. (2015). Investigating 
technical and non-technical factors influencing modern code review. Empirical Software Engineering, 
1-28.

\bibitem{con4} H. Joshi, C. Zhang, S. Ramaswamy, and C. Bayrak. Local and global recency 
weighting approach to bug prediction. In MSR'07, pages 33-34, Minneapolis, Minnesota, USA,
May 2007. IEEE Computer Society.

\bibitem{con5} M.-T. J. Ostrand, F.-E. J. Weyuker, and R. M. Bell. Predicting the location 
and number of faults in large software systems. IEEE Trans. Softw. Eng., 31(4):340-355, 2005.

\bibitem{con6} J. Sliwerski, T. Zimmermann, and A. Zeller. When do changes induce fixes? In 
MSR'05, pages 24-28, Saint Louis, Missouri, USA, May 2005. ACM.

\bibitem{con7} C. Bird, A. Bachmann, E. Aune, J. Duffy, A. Bernstein, V. Filkov, and P. Devanbu. 
Fair and balanced?: bias in bug-fix  datasets. Proceedings of the the 7th joint meeting of the 
European software engineering conference and the ACM SIGSOFT symposium on The foundations of 
software engineering (FSE 09). Amsterdam, The Netherlands: ACM, 2009, pp. 121–130.

\bibitem{con8} G. Antoniol, K. Ayari, M. Di Penta, F. Khomh, and Y.G. Guhneuc. 
Is it a bug or an enhancement?: a text-based approach to classify change requests. 
Proceedings of the 2008 conference of the center for advanced studies on collaborative research: 
meeting of minds (CASCON 08). Ontario, Canada: ACM, 2008, pp. 304–318.

\bibitem{con9} M. Fischer, M. Pinzger, and H. Gall. Analyzing and relating bug report data for 
feature tracking. In WCRE'03, pages 90-99, Victoria, B.C., Canada, November 2003.

\bibitem{con10} D. Cubranic and G. C. Murphy. Hipikat: Recommending pertinent software development 
artifacts. In ICSE'03, pages 408-418, Washington, DC, USA, 2003.

\bibitem{con11}  A. Schroter, T. Zimmermann, R. Premraj, and A. Zeller. If your bug database could 
talk... In ICSE'06, pages 18-20, Rio de Janeiro, Brazil, September 2006.

\bibitem{con12} T. Zimmermann, R. Premraj, and A. Zeller. Predicting defects for eclipse. In PROMISE'07, 
pages 1-9, Minneapolis, Minnesota, USA, May 2007. IEEE Computer Society.

\bibitem{con13} T. Zimmermann and P. Weissgerber. Preprocessing cvs data for fine-grained analysis. In 
MSR'04, pages 2-6, Edinburgh, Scotland, UK, May 2004. ACM.

\bibitem{con14}  G. A. Liebchen and M. Shepperd. Data sets and data quality in software engineering. In 
PROMISE'08, pages 39-44, New York, NY, USA, 2008.

\bibitem{con15} N. Bettenburg, S. Just, A. Schroeter, C. Weiss, R. Premraj, and T. Zimmermann. Quality of 
bug reports in eclipse. In eTX'07, pages 21-25, Montreal, Canada, October 2007.

\bibitem{con16} N. Bettenburg, S. Just, A. Schroeter, C. Weiss, R. Premraj, and T. Zimmermann. 
What makes a good bug report? Technical report, Saarland University, Saarbrucken, Germany, September 2007.

\bibitem{con17} N. Bettenburg, R. Premraj, T. Zimmermann, and S. Kim. Duplicate bug reports considered 
harmful... really? In ICSM'08, pages 337-345, October 2008.

\bibitem{con18} P. Hooimeijer and W. Weimer. Modeling bug report quality. In ASE'07, pages 34-43, Atlanta, 
Georgia, USA, November 2007.

\bibitem{con19} K. Chen, S. R. Schach, L. Yu, J. Outt, and G. Z. Heller. Open-source change logs. Emp. 
Softw. Eng., 9(3):197-210, 2004.

\bibitem{con20} A. Bachmann and A. Bernstein. Software process data quality and characteristics: a 
historical view on open and closed source projects. In IWPSE-Evol'09, pages 119-128, Amsterdam, The 
Netherlands, August 2009.

\bibitem{con21} J. Aranda and G. Venolia. The secret life of bugs: Going past the errors and omissions 
in software repositories. In ICSE'09, pages 298-308, May 2009.

\end{thebibliography}

%%%%%%%%%%%%%%%%%%%%%%%%%%%%%%%%%%%%%%%%%%%%%%%%%%%%%%%%%%%%%%%%%%%%%%

%%%%%%%%%%%%%%%%%%%%%%%%%%%%%%%%%%%%%%%%%%%%%%%%%%%%%%%%%%%%%%%%%%%%%%

\printindex
\end{document}





